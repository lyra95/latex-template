\documentclass[../main.tex]{subfiles}
\begin{document}

\section{Preliminary}
\begin{theorem}
    Cauchy's theorem.
    \begin{equation}
        a_n = \frac{f^{(n)}(0)}{n!} = \frac{1}{2\pi i} \int_{|z|=r} \frac{f(z)}{z^{n+1}} \, dz
    \end{equation}
\end{theorem}

\begin{definition}
    Complex logarithm is defined as
    \begin{equation}
        \log z = \ln r + i\theta \quad (z=re^{i\theta}, \; r>0, \; \theta \in \mathbb{R})
    \end{equation}
    Here $\theta$ is always taken as the principal value, i.e., $-\pi < \theta \leq \pi$.
    Remind that $\log z$ is not continuous on the negative real axis.
\end{definition}

On $z\ll1$
\begin{equation}
    \sum_{n=0}^{\infty} p(n) z^n = \prod_{k=1}^{\infty} \frac{1}{1-z^k}
\end{equation}

\begin{proposition}
    \begin{equation}
        \int \frac{1}{1-w} \, dw = -\log(1-w)
    \end{equation}
    Therefore,
    \begin{equation}
        -\log(1-w) = \int \sum_{n=0}^{\infty} w^n \, dw = \sum_{n=0}^{\infty} \frac{w^{n+1}}{n+1} = \sum_{n=1}^{\infty} \frac{w^n}{n}
    \end{equation}
\end{proposition}

Say $F(z) = \prod_{k=1}^{\infty} \frac{1}{1-z^k}$. Then, using the above proposition and switching the order of summation,
\begin{equation}
    \log F(z) = -\sum_{k=1}^{\infty} \log(1-z^k) = \sum_{k=1}^{\infty} \sum_{j=1}^{\infty} \frac{z^{kj}}{j} = \sum_{j=1}^{\infty} \frac{1}{j} \left( \sum_{k=1}^{\infty} z^{kj} \right) = \sum_{j=1}^{\infty} \frac{1}{j} \cdot \frac{z^j}{1-z^j}
\end{equation}

The set $\{z \ll 1\}$ is equivalent to $\{ w, \Re(w) > 0 \}$ where $z = e^{-w}$. Substituting this,
\begin{equation}
    \log F(e^{-w}) = \sum_{j=1}^{\infty} \frac{1}{j} \cdot \frac{e^{-jw}}{1 - e^{-jw}} = \sum_{j=1}^{\infty} \frac{1}{j} \cdot \frac{1}{e^{jw} - 1}
\end{equation}

\begin{proposition}
    When $w$ is near $0$,
    \begin{equation}
        e^{w} = 1 + w + \frac{w^2}{2} + \frac{w^3}{6} + \cdots
    \end{equation}
    \begin{equation}
        \begin{split}
            \frac{1}{e^{w}-1} & = \frac{1}{\sum\frac{w^n}{n!}} \\
            & = \frac{1}{w} \cdot \frac{1}{1 + \frac{w}{2} + \frac{w^2}{6} + \cdots} \\
            & = \frac{1}{w} \left( 1 - \frac{w}{2} + \frac{w^2}{12} - \frac{w^4}{720} + \cdots \right) \\
            & = \frac{1}{w} - \frac{e^{-w}}{2} + cw + \cdots
        \end{split}
    \end{equation}
\end{proposition}

Let's decompose the term $\frac{1}{e^{jw}-1}$ as follows:
\begin{equation}
    \frac{1}{e^{jw}-1} = \left( \frac{1}{jw} - \frac{e^{-jw}}{2} \right) + \left( \frac{1}{e^{jw}-1} - \frac{1}{jw} + \frac{e^{-jw}}{2} \right)
\end{equation}
Later, we will analysis $w$ near $0$ part and the other part separately. Plugging this into the expression of $\log F(e^{-w})$,
\begin{equation}
    \begin{split}
        \log F(e^{-w})
        &= \sum_{j=1}^{\infty} \frac{1}{j} \left( \frac{1}{jw} - \frac{e^{-jw}}{2} \right)
            + \sum_{j=1}^{\infty} \frac{1}{j} \left( \frac{1}{e^{jw}-1} - \frac{1}{jw} + \frac{e^{-jw}}{2} \right) \\
        &= \frac{\pi^2}{6w}
            - \frac{1}{2} \log(1 - e^{-w})
            + \sum_{j=1}^{\infty} \frac{1}{j} \left( \frac{1}{e^{jw}-1} - \frac{1}{jw} + \frac{e^{-jw}}{2} \right)
    \end{split}
\end{equation}

\end{document}
